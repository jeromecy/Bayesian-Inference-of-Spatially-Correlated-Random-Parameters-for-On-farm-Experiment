\documentclass[a4paper]{article}   	% use "amsart" instead of "article" for AMSLaTeX format
\usepackage{geometry}                		% See geometry.pdf to learn the layout options. There are lots.
\geometry{left=1.5cm,right=1.5cm,top=1.5cm,bottom=1.5cm}                   		
\usepackage{graphicx,subcaption}					
\usepackage{amssymb}
\usepackage{indentfirst}
\usepackage{amsmath}
\usepackage{amsthm}
\usepackage{bm}
\usepackage{lineno}
\usepackage{setspace}
\usepackage{booktabs,multirow}
\usepackage{authblk}


\RequirePackage[colorlinks,citecolor=blue,urlcolor=blue]{hyperref}

\newtheorem{theorem}{Theorem}
\newtheorem{lemma}[theorem]{Lemma}
\newtheorem{corollary}[theorem]{Corollary}
\newcommand{\E}{\mathrm{E}}
\newcommand{\D}{\mathcal{MD}}
\newcommand{\Var}{\mathrm{Var}}
\newcommand{\Cov}{\mathrm{Cov}}
\newcommand{\Corr}{\mathrm{Corr}}
\newcommand{\tr}{\mathrm{tr}}
\newcommand{\R}{\texttt{R}}
\newcommand{\asreml}{\texttt{ASReml-R}}
\newcommand{\brms}{\texttt{brms}}
\newcommand{\rstan}{\texttt{rstan}}
\newcommand{\elpd}{elpd\textsubscript{loo}}
\newcommand{\psis}{elpd\textsubscript{psis-loo}}
\newcommand{\ploo}{p\textsubscript{loo}}
\newcommand{\BigO}[1]{{\rm O}\left(#1\right)}
\newcommand{\eg}{e.g.\ }
\newcommand{\ie}{i.e.\ }
\newcommand{\iid}{\textrm{i.i.d.\ }}
\newcommand{\N}{\mathcal{N}}
\newcommand{\AR}{\mathrm{AR}1}
\newcommand{\Matern}{Mat\'ern }

\newcommand{\qtitle}[1]{\textit{\textbf{#1}}}

\usepackage[url=false, isbn=false, eprint=false, backref=true, style= authoryear, backend=bibtex, maxcitenames=2, giveninits=true, maxbibnames=100]{biblatex}
\addbibresource{BayesOFE.bib}


\title{Response to Reviewers' Comments \\ 
\large on the paper Manuscript Number: FIELD-D-21-01010 \\ Bayesian Inference of Spatially Correlated Random Parameters for On-farm Experiment}


\author[1]{Zhanglong Cao}
\author[1]{Katia Stefanova}
\author[1,2]{Mark Gibberd}
\author[1,3]{Suman Rakshit}

\affil[1]{SAGI West, School of Molecular and Life Sciences, Curtin University, Perth, Australia}
\affil[2]{Centre for Crop and Disease Management, School of Molecular and Life Sciences, Curtin University, Perth, Australia}
\affil[3]{School of Electrical Engineering, Computing, and Mathematical Sciences, Curtin University, Perth, Australia}


\begin{document}
\maketitle


We are all grateful to the reviewers and editors for their time and efforts on our manuscript. The comments from the reviewers are indeed will make the manuscript better in terms of accurate description and easy understanding. 

We have addressed the comments and amended the manuscript accordingly. The amendments in the manuscript to address the reviewers' comments are \textcolor{red}{highlighted in red colour}. We also polished the manuscript to make it clear to read. 


\section*{Reviewer 1}

The authors implement a spatially explicit random coefficient model for data from on-farm trials that allows mapping the optimal levels of an agronomic input. They use a Bayesian framework and illustrate the approach using a randomized complete block trial with six nitrogen treatment levels.

I applaud the authors for implementing this model, which is a major step forward in the analysis of spatially referenced on-farm trial data. I have a couple of comments for the authors.

\begin{enumerate}
    \item \qtitle{Why did they not simply use a REML package to fit the model? If there are any major obstacles, these should be discussed, as many users would use REML as a default approach.}
    
    In Line 489 in the Discussion section, we added ``\textcolor{red}{Another potential method of analysis is based on the residual maximum likelihood (REML). The estimation of regression coefficients under the REML framework would require the development of a computing algorithm that would take into account the spatial correlation of the random effects while computing the best linear unbiased predictors of the treatment effects. }''
    
    The use of the REML package is discussed in our \textit{Random Coefficient Regression} (RCR) paper, which is in preparation. To implement the model in the REML package, such as \asreml, we split the process into two phases: in phase 1, we estimated the correlation parameters $\rho_c$ and $\rho_r$ of $\AR\otimes\AR$ through a two-dimensional grid search; in phase 2, the final model is estimated with the correlation parameters found in phase 1. The entire process is prolonged compared to the Bayesian model in our paper. Additionally, for Bayesian approach, all parameters are estimated at the same time by the NUTS sampler rather than in two separate phases by REML. We compare and discuss the two approaches in the RCR paper. 
    
    The objective of this paper is proposing the Bayesian approach and comparing with GWR. The Bayesian approach in \rstan\ is more efficient and it overcomes the disadvantages of GWR, such as bandwidth selection. 
    
    \item \qtitle{The example is a randomized complete block design with three reps and six nitrogen levels. I would think that this design is superbly inadequate for assessing locally varying optimal input rates. Nitrogen levels would need to be varied on a finer grid, as the authors hint themselves in the introduction. Can they not find a more suitable dataset to illustrate their modelling approach? If not, the paper is best characterized as a proof-of-concept.}
    
	\textcite{Piepho2011Statistical, Pringle2004FieldScale} have stated that to obtain an optimal treatment map, a systematic design over performs to a randomised design. We found that the results are consistent through simulation studies. The manuscript \textit{Optimal design for on-farm strip trials --- systematic or randomised?} is in preparation. If the goal of OFE is obtaining a smooth map showing the optimal level of a controllable input across a grid made by rows and columns covering the whole field, a systematic design should be preferred over a randomised design in terms of robustness and reliability. 
	
	However, the model of obtaining such a treatment map was not properly developed until the GWR paper by \textcite{Rakshit2020Novel}. With the adoption of Bayesian approach, we are comparing the results with GWR on the same data set. Besides, we proposed the Bayesian workflow and discussed potential model misspecification, which are ignored by researchers. 

    \item \qtitle{The paper has a lot of mathematical detail. This is unavoidable with this kind of modelling and an absolute necessity. That being said, it seems to me that the example could play a much more central role in guiding the uninitiated reader. For example, the authors never even state that the purpose of their analysis is to determine the locally varying rate of nitrogen input! This key objective, along with a brief sketch of the example, would sit well in the introduction, thus triggering the readers' interest. Without such a teaser, my prediction is that 99.9\% of the readers will put the paper down after the first few equations. Also, when stating the model for the first time, explain what is in $Zu$!! This is the key component of the model. Moreover, explain that you will be fitting a quadratic polynomial, as this is essential in order to be able to determine an optimum! Also tell readers how this is done, even if it seems too obvious to a statistician: 99.9\% of your readership will not be statisticians.}

    We have introduced our research object in Line 47: ``Our  aim  in  this  paper  is to  obtain  spatially-varying  estimates  of  treatment  effects,  which  in  turn  enables  the creation of spatial maps of optimum treatment levels for large paddocks. ''
    
    For the term $Zu$ in equation (1), we have explained what they are in Line 114. 
    
    For the terms $z$ and $u$ in equation (5), we have explained what they are in Line 145. 
    
    For the term $z_u$ in equation (17), we have explained in Line 220. 
    
    \item (4) \qtitle{The Bayesian $R^2$ in eq. (27) does not account for spatial covariance. It uses just the marginal variance. This seems inappropriate for the data at hand, which are spatially correlated, a feature that is, in fact, central to the whole modelling approach. In linear mixed models in general, and in a spatial context in particular, there are pairwise correlations among observations. A natural way to account for this is the semivariance and averaging this across all pairs of observations, or integrating across the random field, naturally leads to an $R^2$ that does account for correlation. See Piepho, H.P. (2019): A coefficient of determination ($R^2$) for generalized linear mixed models. Biometrical Journal 61, 860-872.}
        
    
    The concept of Bayesian $R^2$ is different from the $R^2$ of LMM or GLMM or ANOVA. Let's say the model
   	\begin{equation}\label{eq:modelmatrix}
    	\bm{Y} = \bm{X}\bm{b}+\bm{Z}\bm{u}+\bm{e}.
    \end{equation}
	For the classical LMM, we may have 
	$\E(\bm{Y}) = \bm{X}\bm{b}$ and $\Var(\bm{Y}) = \bm{Z}\Sigma_u\bm{Z}^\top+\Sigma_e$. The spatial covariance factor is embedded in the term $\Sigma_u$, which is in the ``variance'' term. Then the $R^2$ should be estimated by the method given by \textcite{Piepho2019Coefficient}. 
	
	But for Bayesian approach, we call it a hierarchical model, where $\E(\bm{Y}) = \bm{X}\bm{b} + \bm{Z}\Sigma_u\bm{Z}^\top$ and $\Var(\bm{Y}) = \Sigma_e$. It is because, in Bayesian approach, ``both model components $\bm{b}$ and $\bm{u}$ are treated similarly'' and ``In this way, the uncertainty in the estimates of these model parameters can be easily derived using posterior distributions. '' \parencite{Burkner2017Brms}. 

    Hence, in this paper, the Bayesian $R^2$ given by \textcite{Gelman2019Rsquared} is appropriate. 
    
    \item \qtitle{The field layout of the treatment design should be shown. Were treatments randomized within complete reps? How many observations were there per plot? What were the six treatment levels? This is absolutely key information, without which it is impossible to judge the results of the analysis. For example, it is unclear that the predicted optima of nitrogen input are within the observed range of levels.}

    Added line 307 that ``The data were produced by a yield monitor in an Argentinian corn field trial conducted by incorporating six nitrogen rate treatments \textcolor{red}{which are systematically allocated} in three replicated blocks comprising eighteen strips \textcolor{red}{(ranges) and 93 rows}. ''
    
    We also replace Figure 1 (c) with the allocation of the treatments. 
    
    \item \qtitle{It would be useful to show a model with fixed regression terms, i.e. $V_u = 0$, as a benchmark, as this is what would be routinely used for this kind of experiment. In fact, showing that $V_u$ is not zero would be central in order to demonstrate that it is worthwhile to fit spatially varying coefficients and that the central hypothesis of precision farming is valid (see Piepho et al. 2011, whom the authors are citing).}

    Our model didn't include the conventional fixed and random terms, such as the replicate structure and blocking structure. It is because the replicate factor is not significant. 
    
    Alternatively, we use nitrogen treatment levels as both fixed and random terms. Our assumption is that there is a global trend on the treatment against yield, and the local treatment in each grid is adjusted by the model. The main purpose of the proposed approach is to compare with GWR model \parencite{Rakshit2020Novel}.

    \item \qtitle{How did the authors determine the optimal inputs? Some explanation of the algebra would be useful. I suspect these are inputs maximizing yield, but as a farmer I'd be more concerned with economic optima. What about credible intervals for the predicted optima? How wide are they? Can this width be mapped as well?}

    Added: Line 315 ``\textcolor{red}{To obtain the map of locally varying optimal input rates, we specified a quadratic regression model, in which the corn yield is modelled as a quadratic function of the nitrogen rate. The optimal treatment can be determined by estimating the coefficients of the quadratic regression model at each grid point.} ''
    
    Also revised from Line 436 ``\textcolor{red}{Because we have fitted a quadratic response of yield to nitrogen rates, we can compute the optimal nitrogen rate $\tilde{N}_{i}$ for the $i$th grid point using $\tilde{N}_{i} = -\hat{\beta}_{1}/(2\hat{\beta}_{2})$, $i=1,\ldots,n$. However, if the optimum rate exceeds the maximum rate $N_{\mbox{max}} = 124.6$ kg/ha used in the trial, the maximum rate has been chosen as the optimal rate. Therefore, we can compute the adjusted optimal rate $\hat{N}_i = \min\{ \tilde{N}_i, N_{\mbox{max}}\}$ for $i=1,\ldots,n$. }''

    \item \qtitle{How was the topographic factor incorporated into the model, and how does this factor align with blocks? It would help to see the estimates of all the fixed effects of the models fitted.}

    The block/replicate factor is not significant compared to the topographic factor. But when we tried to incorporate the topographic factors as either fixed and random term in our study, and the posterior checking shows severe model misspecification even though the $R^2$ value is high. Particularly when topographic factor is fitted as the random term, the model is be over fitted because the spatial variance is fitted twice. 
    
    Being more ambitious, we may try to fit $\Sigma_u(topo)$ varying across different topographies. But the model will be more complex and slows down the computing speed. 

    \item \qtitle{Eq. (11): What does LKJ stand for? What are the parameters of this model? I did not find and output for the actual correlations or covariances among the random coefficients. Can these be reported? Random coefficient models are notoriously difficult to fit, and proper scaling of the covariates is usually essential. Did the authors encounter such issues and how did they deal with them?}

    Revised the paragraph in line 192: ``\textcolor{red}{For the matrix $R_u$ with correlation coefficients, we specify the Lewandowski-Kurowicka-Joe (LKJ) distribution \parencite{Lewandowski2009Generating} as the prior distribution, and this specification is given by
	\begin{equation}\label{eq:RPrior}
		R_u \sim \text{LKJcorr}(\epsilon),
	\end{equation}
	where $\text{LKJcorr}(\epsilon)$ is a positive definite correlation matrix sampled from the LKJ distribution that depends on the value of a positive parameter $\epsilon$. The parameter $\epsilon$ controls the correlations in a way that, as the value of $\epsilon$ increases, the correlations amongst parameters decrease.}''
    
    As shown in Figure 2. The correlation parameters are reported in Table 5.
     

    \item \qtitle{Eq. (17): The $\AR\otimes \AR$ model vor $V_s$ has different parameters than the one for $\Sigma_e$. Somehow, this should be reflected in the notation of both model components.}

    In conventional statistical model in equation (3-5), the covariance matrix is imposed to $\Sigma_e$. But for the proposed Bayesian model, the covariance matrix is incorporated with the random parameter $u$. So for $\Sigma_e$, it is just $\sigma_e*I$. 

    \item \qtitle{Figure 8: This map only shows optimum levels for a very limited part of the field. What about the big white area??!!}

    This proves that the quadratic relationship is not significant for the data set we have. It is more like a linear relationship. The results are consistent with GWR, where the adjusted $p$-values are more than 0.05 for the quadratic term.

    \item \qtitle{L405: I do not think it is correct to say that the influence of the prior is washed out if the model is good enough. It is the amount of data that determines the influence of the prior.}
    
    Thank you for pointing it out. It is not a precise statement. We have made the amendments accordingly. 
    
    Line 271: The application of posterior predictive distributions is robust to prior specification because the details of the prior are washed out by the likelihood \parencite{Gelman2017Prior}.
    
    Revised line 453: However, the influence of the prior \textcolor{red}{reduces if the amount of data increases}.    
    
\end{enumerate}





\section*{Reviewer 2}

The manuscript deals with an important topic of modelling spatial variability in large on-farm trials. A Bayesian framework is adopted to estimate the posterior distribution of parameters. Also, the proposed method is applied on a real on-farm strip trial from Las Rosas, Argentina, with the aim of obtaining a spatial map of optimal nitrogen rates for the entire paddock. The manuscript requires revision before it can be accepted for publication. My specific comments are listed below:
\begin{enumerate}
    \item \qtitle{Why was weakly informative prior preferred? How do you define a weakly or strongly informative prior?}
    
    The general idea is that such a prior affects the information in the likelihood as weakly as possible \parencite{Gelman2017Prior}. However, vague priors failed in the prior checking. Based on \textcite{Gabry2019Visualization}'s work, we say that a prior leads to a weakly informative joint prior data-generating process if draws from the prior data-generating distribution could represent any data set that could plausibly be $p(y)$ observed. A strongly informative prior might have a smaller deviation and/or high influence on the posterior distribution.
    
    \item \qtitle{Four models are used in the analysis with conditions of with or without spatial correlation. Why was the model uncertainty of each of them not characterised?}
    
    The uncertainty and misspecification are investigated and determined by LOO PIT and Pareto $\hat{k}$ values \parencite{Gabry2019Visualization}. It is identified that the model is misspecified if the spatial variability is not accounted for. 

    \item \qtitle{Figure 3 presents the realisations of 100 simulations. It is known that more informative prior will give better results than vague prior. The results do not suggest anything new. The authors could have analysed the uncertainty of model prediction.}
    
    It is not new, but it provides a visualisation perspective on prior checking, rather than picking up priors from references and evaluating them by posterior checking. 
    
    \item \qtitle{What is expected error variance in equation (27)? How to determine it? Will the error in spatial variability problem be linear?}
    
	The term ``error variance'' maybe confusing. Alternatively, we use the term ``residual variance''. 
	
	Added in Line 290: \textcolor{red}{residual variance}. Reference \textcite{Gabry2019Visualization}. 
	
	For a Bayesian hierarchical model, we have $\E(\bm{Y}) = \bm{X}\bm{b} + \bm{Z}\Sigma_u\bm{Z}^\top$ and $\Var(\bm{Y}) = \Sigma_e$. The spatial variability is accounted for in the random term rather than in the error term. 
    
    \item \qtitle{In equation (28) how are correlation parameters determined?}
    
    Example in figure 2. It is a positive definite correlation matrix sampled from the Lewandowski-Kurowicka-Joe (LKJ) distribution \parencite{Lewandowski2009Generating, McElreath2015Statistical}. 
    
    \item \qtitle{In Figure 4, PP(posterior predictive) checking is done against the observed data Y. But you have used the same dataset to update the parameters and obtained posterior distribution of parameters. Hence, the posterior predictive results will be close to the observations. The authors could have checked the reliability of the model by performing PP checking on some other dataset obtained from the same site.}
    
    The PP check is like a simulation checking method that uses posterior distribution to generates ``new'' data and compare it with the observed data set. It is used to check the performance of the parameters by comparing two data. It is defined in equation (24). 
    
    
\end{enumerate}

\renewcommand\bibname{References}% change bibliography title to 
\addtocontents{toc}{Bibliography}
\printbibliography


\end{document}